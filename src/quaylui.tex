\begin{center}
    \textbf{\LARGE Buổi 0+1: Quay lui}
\end{center}

\begin{tabular}{|p{5mm}|p{4cm}|p{4cm}|p{4cm}|p{1.5cm}|}
    \hline
    TT & Nội dung & \multicolumn{2}{c|}{Hoạt động dạy học} & Thời gian\cr
    \cline{3-4}
     &  & Hoạt động của người dạy & Hoạt động của người học & \cr\hline
    1 & \textbf{Dẫn nhập}\newline
    Giới thiệu về môn Cấu trúc Dữ liệu và Giải thuật.\newline
    Chọn cấu trúc dữ liệu \& thiết kế giải thuật.\newline
    Độ phức tạp tính toán.\newline
    Ngăn xếp, đệ quy và quay lui (như trên sách).
    & & &5 phút\newline 20 phút\newline\cr
    \hline
    2 & \textbf{Giảng bài mới}\newline
    Mã giả của thuật toán quay lui.
    & & &15 phút\cr
    \hline
    3 & \textbf{Luyện tập} &
    Đối với bài toán phân tích số và bài toán xếp hậu, cho người học 15 phút mỗi bài suy nghĩ và cài đặt thử.
    &
    Liệt kê dãy nhị phân\newline
    Liệt kê các tổ hợp chập k\newline
    Bài toán phân tích số\newline
    Bài toán xếp hậu
    &15 phút\newline 15 phút\newline 45 phút\newline 45 phút\cr
    \hline
    4 & \textbf{Hướng dẫn tự học} & \multicolumn{2}{c|}{Tham khảo thêm ở \href{https://oj.vnoi.info/problems/?type=3&point_start=&point_end=&order=points}{VNOJ} hay \href{https://cses.fi/problemset/}{CSES Problem Set}.} & 10 phút\cr
    \hline
\end{tabular}